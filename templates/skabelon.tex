\documentclass{ucph-revy}

\usepackage[utf8]{inputenc}
\usepackage[OT1]{fontenc}
\usepackage[danish]{babel}
\usepackage{hyperref}
\urlstyle{sf}

\version{1.0}
%% følgende definitioner kan godt fjernes, hvis der ikke er brug for dem:
\revyname{<+REVUENAME+>}
\revyyear{<+REVUEYEAR+>}
\status{eksempel}
\eta{$1$ minut, $47$ sekunder}
\responsible{Dig}

\title{Eksempel}
\author{en eksempelsmed}
\melody{Queen: „Bohemian Rhapsody`` (\url{https://youtu.be/fJ9rUzIMcZQ})}

\begin{document}
\maketitle

\begin{roles}
  \role{PH}[Jophiel] Per Hedegaard
  \role{I} Vred instruktør
  \role{A9}[A--ni] Anine
\end{roles}
%% props-miljøet kan fjernes, hvis der ikke er brug for det
\begin{props}
  \prop{Hedegaard--hår}[Person, der skaffer]
  \prop{Didgeridoo}[Anine]
\end{props}

\begin{sketch}
  \scene{Lys op}
  \says{PH} Hej venner. I den her sketch, der spiller jeg Per
  Hedegaard. Det kan man se ud fra farven på mit hår. Men bare rolig,
  jeg er ikke Per Hedegaard i virkeligheden, der er ikke nogen
  regneøvelser, I har glemt at lave.

  Se, hvis jeg tager håret af \act{fjerner parykken}, så er det helt
  tydeligt, at jeg ikke er den rigtige Hedegaard.

  \says{I}[Stormer ind på scenen] \emph{NEEEEEJ}, for helvede. Du må
  ikke pille ved den fjerde væg. Det er simpelthen så uprofessionelt,
  det der!

  \says{PH} Se, venner, det var meningen, at der skulle have været en
  anden replik der, men fordi der ikke er sat nogen på den rolle i
  rollelisten, så kunne vi ikke høre nogen som helst kritik af min
  performance.

  \does{I} reagerer med åbenmundet vantro.

  \says{PH} Det må også være svært at følge skuepil--instruktioner,
  når man ikke eksisterer, kan jeg forestille mig.

  \says{A9} Hej Jophiel.
  \does{PH} tager sin paryk på igen.
  \says{A9} Hej, Per Hedegaard.
  \says{A9} Var det ikke meningen, at det her skulle være en sang?
  \says{PH} Jo, for dælen. Har du en didgeridoo med?
  \says{A9} Didgeridoo?
  \says{PH} Ja. Der står i rekvisitlisten, at du skaffer en.
  \says{A9} Jamen så må det jo passe.
  \says{PH} Ja, ellers skriver vi "`Person, der skaffer,"' for at
  minde os om at skrive et navn ind.
  \says{A9} Nå, men så må jeg jo hellere begynde at lede. \act{Vender
  sig mod Bandet} Bandet, har I...?
  \scene{Hvert bandmedlem holder sin egen didgeridoo op.}
  \says{A9}[Til PH] Den ser ud til at være i vinkel.
  \says{PH} Sejt.
  \says{PH}[Råber til Bandet] Og \emph{tre!} \emph{Fire!}
\end{sketch}
\begin{song}%
  \scene{Bandet spiller ``Bohemian Rhapsody" af Queen, udelukkende på
    didgeridoo.}

  \sings{A9} Åh, Per Hede
             Er den lede
             Og han dumped' mig i kvant!

  \sings{PH} Per Hedegaard har er reeksamenssæt til dig
  \sings{A9} Til mig?
  \sings{PH} Til dig!
  \scene{Episk didgeridoo--solo, mens A9 løser reeksamenssæt.}

  \scene ...

  \does{A9} rækker hånden i vejret, og bandet stopper.
\end{song}%
\begin{sketch}
  \says{PH}[Er et stykke tid om at lægge mærke til A9, men kommer til
  sidst over] Ja, spørgsmål?
  \says{A9} Det her eksamenssæt er jo trivielt...?
  \scene{Lys ned}
\end{sketch}

\end{document}
\endinput
%%
%% End of file `skabelon.tex'.
