\documentclass{ucph-revy}
\usepackage[OT1]{fontenc}
\usepackage[utf8]{inputenc}
\usepackage[danish]{babel}
\usepackage{tikz}
\usepackage[a1paper,lmargin=0.5cm,rmargin=0.5cm,bmargin=1cm]{geometry}
\usepackage{longtable, array, multirow}

\newlength{\height}
\newcommand{\actor}[1]{\rotatebox{90}{#1\ }}

\newcommand{\setheight}[1]{\setlength{\height}{#1}}
\setheight{1.5em}
\newcommand{\onstage}[1]{%
  \tikz[baseline=(mid.south)]{
    \draw (-.5em,0) -- (.5em,0) (-.5em,#1) -- +(1em,0);
    \draw[ultra thick] (-.2em,0) -- +(0,#1)
    (.2em,0) -- +(0,#1) coordinate[pos=.5] (mid);
  }%
}
\newcommand{\offstage}[1]{\tikz[baseline=(mid)] \draw[gray,dashed] (0,.2em) -- (0,#1) coordinate[pos=.5] (mid);}
\renewcommand{\arraystretch}{.8}

\title{Tidsdiagram}
\version{<+VERSION+>}

\begin{document}
\maketitle

\begin{longtable}{p{3em}r*{<+NACTORS+>}{c}}
  \hline
  &Nummer / Person&<+ACTORS+>\\
  \hline
  \endhead
  \hline
  \endfoot
  % &\multicolumn{39}{l}{
  %   \tikz[remember picture,overlay] \coordinate (st) at (0,.166em);
  %   \hfill
  %   \tikz[remember picture,overlay] \draw (st) rectangle (0,0);
  % }\\
  <+NUMBERS+>
\end{longtable}

\end{document}
%%% Local Variables:
%%% mode: LaTeX
%%% TeX-master: t
%%% End:
